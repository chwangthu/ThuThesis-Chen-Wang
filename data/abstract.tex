% !TeX root = ../main.tex

% 中英文摘要和关键字

\begin{abstract}
平板电脑、智能手表等电子设备因为其方便易用性已经成 为了人们生活和移动办公的重要组成部分。然而,在这些移动设备上的文本输入仍然存在在尺寸小、误触多并且输入慢等问题,另一方面,手机的硬件水平和传感能力在不断提高,为新的交互方式提供了可能性。
  
  基于手机的多种传感信号,我们开发了一个能在任意桌面上进行文本输入的技术。我们首先通过声音信号识别出用户的点击事件,然后通过深度图像获取点击的三维坐标。基于上述的点击识别,我们进行了用户实验采集用户在桌面上的输入数据,分析了用户的落点分布、输入速度、键盘大小等特征。然后,我们对点击数据进行了一系列的模拟实验,尝试不同的算法的有效性。最后,我们选取了半键盘相对模型进行评测,结果表示用户输入速度能够达到XXWPM,字符级别错误率为XXX。此外,我们还在系统中集成了字符级别输入,并对字符级别输入和单词与字符混合输入方式进行了测试。

  % 关键词用“英文逗号”分隔
  \thusetup{
    keywords = {文本输入,手机传感},
  }
\end{abstract}

\begin{abstract*}

  \thusetup{
    keywords* = {Text Entry, Phone Sensing},
  }
\end{abstract*}
