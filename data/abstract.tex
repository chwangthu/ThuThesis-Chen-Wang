% !TeX root = ../main.tex

% 中英文摘要和关键字

\begin{abstract}
  近年来,随着科技的发展的发展,平板电脑、智能手表等电子设备已经成 为了人们生活和移动办公的重要组成部分。然而,在这些移动设备上的文本输入仍然存在在反馈差、误触多并且输入慢等问题,另一方面,手机的硬件水平在不断提高,为新的交互方式提供了可能性。
  
  基于手机的多种传感信号,我们开发了一个能在任意桌面上进行文本输入的技术。我们首先通过声音信号识别出用户的点击事件,然后通过深度图像获取点击的三维坐标。基于上述的点击识别,我们进行了用户实验采集用户在桌面上的输入数据,分析了用户的落点分布、输入速度等特征。然后,我们对数据进行了一系列的模拟实验,尝试不同的算法的有效性。最后,我们选取了XX算法进行评测,并且在有无反馈、显示延迟的情况下进行对比,结果表示用户输入速度能够达到XXWPM,字符级别错误率为XXX。

  % 关键词用“英文逗号”分隔
  \thusetup{
    keywords = {文本输入,手机传感},
  }
\end{abstract}

\begin{abstract*}
  An abstract of a dissertation is a summary and extraction of research work
  and contributions. Included in an abstract should be description of research
  topic and research objective, brief introduction to methodology and research
  process, and summarization of conclusion and contributions of the
  research. An abstract should be characterized by independence and clarity and
  carry identical information with the dissertation. It should be such that the
  general idea and major contributions of the dissertation are conveyed without
  reading the dissertation.

  An abstract should be concise and to the point. It is a misunderstanding to
  make an abstract an outline of the dissertation and words ``the first
  chapter'', ``the second chapter'' and the like should be avoided in the
  abstract.

  Key words are terms used in a dissertation for indexing, reflecting core
  information of the dissertation. An abstract may contain a maximum of 5 key
  words, with semi-colons used in between to separate one another.
  \thusetup{
    keywords* = {Text Entry, Phone Sensing},
  }
\end{abstract*}
