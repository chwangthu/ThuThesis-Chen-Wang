% !TeX root = ../main.tex

% 中英文摘要和关键字

\begin{abstract}
平板电脑、智能手表等电子设备因为其方便易用性已经成 为了人们生活和移动办公的重要组成部分。然而,在这些移动设备上的文本输入仍然存在尺寸小、误触多并且输入慢等问题,另一方面,手机的硬件水平和传感能力在不断提高,为新的交互方式提供了可能性。
  
基于手机的多种传感信号,我们开发了一个能在任意桌面上进行高效文本输入的技术。我们首先通过声音信号识别出用户的点击事件,然后通过深度图像计算点击的三维坐标。基于上述的点击识别,我们首先进行了实验采集用户在桌面上的输入数据,分析了用户的落点分布、输入速度、键盘大小等特征。然后,我们对点击数据进行了一系列的模拟实验,尝试不同的算法的有效性。最后,我们选取了半键盘相对模型进行作为单词纠错方法。此外,我们还在系统中集成了字符级别输入。最终,我们对单词级别输入、字符级别输入和单词与字符混合输入方式进行了测试。结果表明,用户输入单词的速度超过40WPM,输入字符的速度超过5WPM。另外,我们也讨论了本输入技术的优势和使用范畴。

本工作的主要贡献如下:
\begin{itemize}
    \item 利用手机的传感能力实现了在任意桌面上十指输入的能力,且使用简单便捷,学习成本低
    \item 技术同时支持单词级别和字符级别输入,并且方便切换,具有广泛的使用空间和扩展空间
\end{itemize}

  % 关键词用“英文逗号”分隔
\thusetup{
    keywords = {十指盲打, 文本输入, 手机传感},
  }
\end{abstract}

\begin{abstract*}
    Electronic devices such as laptops and smartwatches, which are portable and convenient, had been an important part of people's mobile working. However, text input in these devices suffers from problems including limited screen size, unintended touch and slow speed. On the other hand, the hardware and sensing technologies of mobile phones are constantly improving, opening up possibilities of new interactions.
    
    Based on several signals of phones, we developed an efficient text input technique that can be used on any table. We first recognized users touch through sound, then we obtained the 3-dimension coordinate with the aid of depth image. With aforementioned touch detection, we conducted an experiment to collect user typing data on tables and analyzed characteristics such as point distribution, input speed and keyboard size. Then, we performed a series of simulation on typing data and tried to explore the effectiveness of different algorithms. We chose relative model with half keyboard as our decoding algorithm. Moreover, we incorporated character level input in our system. Finally, we evaluated our technique in word level input, character level input and hybrid input. Results showed that users could type more than 40WPM when entering words and 
   type more than 5WPM when entering characters. In addition, we also discussed the advantages of our technique as well as its possible using situations.
    
    The contributions of this paper can be concluded as follows:
    \begin{itemize}
        \item We utilized the sensing abilities of phones to implement text input on any table. It is easy to use and learn.
        \item Our technique can seamlessly switch word level and character level input, which can be widely used and extended to many situations
    \end{itemize}
  \thusetup{
    keywords* = {Ten-finger Typing, Text Entry, Phone Sensing},
  }
\end{abstract*}
