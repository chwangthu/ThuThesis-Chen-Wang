% !TeX root = ../main.tex

% 中英文摘要和关键字

\begin{abstract}
平板电脑、智能手表等电子设备因为其方便易用性已经成 为了人们生活和移动办公的重要组成部分。然而,在这些移动设备上的文本输入仍然存在尺寸小、误触多并且输入慢等问题,另一方面,手机的硬件水平和传感能力在不断提高,为新的交互方式提供了可能性。
  
基于手机现有的传感器,本文研究了一种支持用户在任意桌面上进行高效文本输入的方法。首先通过实验采集了用户在桌面上进行双手盲打时的数据,并基于此分析了用户在落点分布、输入速度、键盘大小等方面的输入行为特征。结果表明,用户输入习惯类似于触摸屏上十指盲打,但是按键更大。基于此结果,本文提出了一种针对任意桌面双手盲打的技术。该技术通过声音信号识别出用户的点击行为,并通过深度图像计算点击的三维坐标,从而实现击键位置的识别。本文进一步通过模拟比较了绝对模型、区分双手和不区分双手的相对模型等不同的输入意图识别方法的性能,结果表明,区分双手的相对模型预测准确性最高,在单人情况下Top-1准确率达到了92.13\%。最后,本文将其作为输入单词的意图推理算法,并结合了字符级别输入功能,实现了支持任意桌面双手盲打的输入方法,并通过用户实验,对单词级别输入、字符级别输入和单词与字符混合三种输入方式进行了测试。结果表明,用户输入单词的速度超过40WPM,输入字符的速度超过5WPM,并且错误率维持在较低的水平。另外,本文还讨论了本输入方法的优势和潜在的使用范畴。

本工作的主要贡献如下:
\begin{itemize}
    \item 通过实验建模了用户在任意桌面上进行双手盲打的输入行为,分析了速度、键盘形状、输入习惯等特征;
    \item 提出了利用手机现有的传感能力实现任意桌面上双手盲打的方法,可支持单词和字符级别的高效准确输入;
    \item 通过用户实验评测了该技术在实际应用场景中的输入性能,并比较了字符和单词级别输入的效果。
\end{itemize}

  % 关键词用“英文逗号”分隔
\thusetup{
    keywords = {十指盲打, 文本输入, 手机传感},
  }
\end{abstract}

\begin{abstract*}
    Portable and convenient electronic devices such as laptops and smartwatches are essential parts for mobile working. However, text input in these devices suffers from problems including limited screen size, unintended touch and slow speed. On the other hand, the hardware and sensing technologies of mobile phones are constantly improving, opening up possibilities of new interactions.
    
    With several signals of phones, we developed an efficient text input technique that can be used on any table. First, we collected users' ten-finger touch typing data on tables through experiments based on which we analyzed characteristics such as touchpoint distribution, input speed and keyboard size, which is similar to typing on touch screens, except a larger key size. Then, we proposed a technique for ten-finger typing on any table, which recognized users' touch and calculated coordinate through sound and depth image. We continued to perform a series of simulations and tried to explore algorithms including absolute model, relative model, etc. Results showed that relative model with half keyboard was the one with highest accuracy, achieving 92.13\% for Top-1. Finally, we chose it as our decoding algorithm for word input and incorporated character input into our system. We evaluated our technique in word-level input, character-level input and hybrid input. Results showed that users could type more than 40WPM for words and type more than 5WPM for characters while remaining a low error rate. We also discussed the advantages of our technique as well as its potential using scenarios.
    
    The contributions of this paper can be concluded as follows:
    \begin{itemize}
        \item We modeled users' ten-finger touch typing behavior on tables through experiment, analyzing features including speed, touchpoints and input preference
        \item We proposed a technique that enabled ten-finger typing on any table, which supported efficient input in both character-level and word-level
        \item We evaluated our technique in real tasks through experiment and compared the results of word-level input and character-level input.
    \end{itemize}
  \thusetup{
    keywords* = {Ten-finger Typing, Text Entry, Phone Sensing},
  }
\end{abstract*}
