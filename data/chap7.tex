% !TeX root = ../main.tex
\chapter{未来工作和总结}
\label{cha:conclusion}
\section{讨论}
根据调研,我们的工作是第一个实现了在不显示键盘的情况下,仅凭借手机的传感信号和用户十指输入的肌肉记忆,能够在任意桌面上进行输入的技术。尽管我们的研究基于iPhone 11,但是信号的处理和算法等可以扩展到任意具备相应传感功能的手机和其他智能设备上,具有广泛的迁移空间。我们的输入技术不仅具备诸多优势,也为未来各种场景下的输入提供了广泛的扩展空间。我们将在本节中作出讨论。

首先,我们的交互模式在新颖的同时保持了简单易用性。整体上输入方式类似于物理键盘,用户能够较容易将自己的肌肉记忆迁移过来。同时,技术满足便携性的要求,用户只需要借助手机和任何一张桌面,无需随身携带任何其他的设备。这些优势为能够充分满足移动办公的需求。

其次,我们的技术能够应用到各种间接输入的场景中,尽管本文的实验将手机屏幕作为显示区域,但是显示的部分完全可以摆脱屏幕限制。将手机只作为接受传感信息的通道,输入内容能够显示在AR或VR眼镜中、智能电视上等等,且比目前在这些设备上输入的方式更加高效方便。

另外,我们的技术同时结合了单词输入和字符输入的功能,并且十分容易切换,大大地拓宽了用户的使用场景。尽管目前的字符级别输入还不够高效,但是我们相信通过借助更大的显示区域和更高效的设计,字符输入效率能够得到进一步提高。

\section{未来工作} %limitation: 声音、图像机器学习、插入删除、支持更好的休息、更多的人
尽管本工作具有如上所述的诸多优势,但是目前的研究内容仍有一些局限性,可以做进一步扩展作为未来工作。

我们相信,克服了这些局限性后,我们的技术能够更好的满足用户的需求,真正走进日常生活的使用范畴。

\section{文章总结}
