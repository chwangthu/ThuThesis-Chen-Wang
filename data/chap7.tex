% !TeX root = ../main.tex
\chapter{未来工作和总结}
\label{cha:conclusion}
\section{讨论}
根据调研,我们的工作是第一个实现了在不显示键盘的情况下,仅凭借手机的传感信号和用户十指输入的肌肉记忆,能够在任意桌面上进行输入的技术。尽管我们的研究基于iPhone 11,但是信号的处理和算法等可以扩展到任意具备相应传感功能的手机和其他智能设备上,具有广泛的迁移空间。我们的输入技术不仅具备诸多优势,也为未来各种场景下的输入提供了广泛的扩展空间。我们将在本节中作出讨论。

首先,我们的交互模式在新颖的同时保持了简单易用性。整体上输入方式类似于物理键盘,用户能够较容易将自己的肌肉记忆迁移过来。同时,技术满足便携性的要求,用户只需要借助手机和任何一张桌面,无需随身携带任何其他的设备。这些优势为能够充分满足移动办公的需求。

其次,我们的技术能够应用到各种间接输入的场景中,尽管本文的实验将手机屏幕作为显示区域,但是显示的部分完全可以摆脱屏幕限制。将手机只作为接受传感信息的通道,输入内容能够显示在AR或VR眼镜中、智能电视上等等,且比目前在这些设备上输入的方式更加高效方便。

另外,我们的技术同时结合了单词输入和字符输入的功能,并且十分容易切换,大大地拓宽了用户的使用场景。尽管目前的字符级别输入还不够高效,但是我们相信通过借助更大的显示区域和更高效的设计,字符输入效率能够得到进一步提高。

\section{未来工作}
尽管本工作具有如上所述的诸多优势,但是目前的研究内容仍有一些局限性,可以做进一步扩展作为未来工作。

首先,目前点击识别是按照声音的响度进行判断,可能会收到外界说话声等干扰,限制了我们的技术需要在安静处使用。使用音频处理的方法进行滤波,能够克服环境噪音的干扰,扩大使用范围。

其次,现有工作已经证明了使用深度学习能够高效识别人的手势\cite{MolchanovGKK15}。在我们工作的基础上收集更多的用户数据,利用深度图像作为输入,不仅能够更好的做到用户手部的追踪,实现更精准的点击位置识别,还可以加入更多更复杂的手势提升用户体验。

另外,现在的单词级别纠错算法较为简单,只考虑了单词长度和用户点击数目相同的情况,对于用户多输入或者漏输入的情况缺乏考虑。相比于触摸屏上的十指盲打,第一个单词为目标单词的准确率还有所差距,并且使用的词库处于较小的水平。此外,目前的输入方式需要用户手臂悬空,容易造成疲劳,如果能够将双手放下来进行输入则更自然舒适。

我们相信,克服了这些局限性后,我们的输入技术能够更好的满足用户的需求,真正走进日常生活的使用范畴。

\section{文章总结}
本文基于手机的声音、深度摄像头多种传感器,实现了一个能够在任何桌面上进行单词和字符级别输入的技术。我们首先从落点分布、按键大小等方面系统研究了用户在桌面上输入的习惯。其次,基于数据模拟探究了不同输入纠错算法的效果。然后,文字介绍了整个系统的设计,包括如何进行字符输入及不同不同模式间的切换。同时,我们对单词、字符、混合三种输入模式进行了用户评测,结果表明输入速度和准确性都取得了较好效果。最后,我们讨论了本输入技术的优势和未来的使用范畴,并指出了进一步的工作方向。