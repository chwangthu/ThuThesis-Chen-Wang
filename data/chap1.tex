% !TeX root = ../main.tex
\chapter{引言}
\label{cha:intro}
\section{研究背景}
近年来,随着科学技术的飞速发展,平板电脑、智能手表等电子设备已经成为了人们生活和移动办公的重要组成部分。统计数据显示,仅2019年第三季度,全球的平板电脑出货量为3760万台\footnote{数据来源:IDC (International Data Corporation), https://www.idc.com/getdoc.jsp?containerId=prUS45620419},智能手表出货量为1420万台\footnote{数据来源:Statista, https://statista.com/statistics/524806/global-smartwatch-vendor-shipments},并且这几年该项数据一直在处于上涨的态势。此外,一些新的可穿戴设备,例如AR眼镜等也吸引着越来越多的用户,有望逐步进入普通人的生活。

文本输入是人们在使用这些智能设备时最为基础的需求之一,自然高效的输入方式能够给生活带来巨大的便捷,极大提升生活的质量。然而,由于种种原因,现有的输入方法仍然难以满足用户的需求,智能设备上的文本输入面临着种种挑战,以下是使用者经常遇到的问题:
\begin{itemize}
    \item \textbf{缺乏触觉反馈}:智能设备上的文本输入由于设备本身的原因,无法像在物理键盘上有触觉反馈。不仅无法在正常按键时有按压的感觉,更难以迅速准确地找到键盘上的Home Row(物理键盘上F键和J键有凸起)\cite{flatglass2011findlater}。反馈的缺乏容易给用户造成疑惑、疲劳等感觉。
    \item \textbf{误触}:现有智能设备的输入绝大部分使用触摸屏作为交互途径。在触摸屏上,相较于物理键盘,将无意的点击例如手指的剐蹭等错误识别的可能性更大。另外,触摸的交互方式决定了用户无法在像在物理键盘上一样在输入间隙将手指或者手掌放置在键盘上休息。
    \item \textbf{点击不准确}:当输入区域较小时,用户的输入方式会从十指转变为单手甚至单指,手指的减少意味着同一手指需要点击更多的按键,给用户自身对于按键的区分带来了挑战。如果在区域非常受限的情况下例如智能手表上的输入,传统QWERTY布局会存在严重的“胖手指”问题,即用户的手指大于按键的大小,导致很难分别邻近的按键。
    \item \textbf{纠错能力不足}:尽管有众多相关研究,但是目前实际产品上的输入纠错能力都比较简单,很少有针对使用者本人习惯、针对不同使用场景的高效算法。
\end{itemize}

有多项研究试图解决上述的问题的某些方面,例如探究更合适的布局\cite{2019tiptext}\cite{2011li1line},尝试新的交互方法\cite{2018forceboard},或者是根据行为特征改进算法\cite{palmboard2020}\cite{2018shitoast}等。

尽管这些工作取得了众多突破,但是有很多技术都依赖于特定的平台,从而停留在原型阶段,难以迁移到人们常用的设备中。此外,用户一般还需要一定的学习成本才能熟练使用。另一方面,电子工艺的突破使得现在的移动设备在计算能力和传感能力上都取得了重大的突破,在此基础上,新的交互方式成为可能。考虑到十指盲打是大多数人最为熟练并且速度最快的打字方式,如果能够让使用者在不随身携带物理键盘的方式下方便使用该方式进行输入,不仅能够大大降低学习成本,而且输入的速度也能够得到保障。本文研究的内容即是借助人们随身携带的手机实现在任意桌面上进行十指输入的技术。

\section{工作简介和贡献}
本工作借助的主要硬件平台是iPhone 11,但是研究的思路和内容可以迁移到任何一台带有前置深度摄像头的移动设备上。工作的主要思路是将手机放置在用户前方一定距离处,借助深度数据和其他信号捕捉用户手部动作,当用户发生敲击动作时进行识别并且计算出具体的坐标。通过用户点击的数据,本文尝试了不同的点击模型,辅以语言模型去推测出用户的输入内容。最终本文实现了相应的单词输入和字符输入的技术,并在实际使用场景下进行了用户评测。

基于前文的工作简介,本文的贡献可以概括成如下内容:
\begin{itemize}
    \item 开辟了一种全新的输入模式,摆脱了具体设备的限制,彻底解决了因为设备造成的尺寸问题。用户可以将手机放置在合适的位置,并且在手机前方足够大的空间内进行自由的十指输入
    \item 用户使用本技术的学习成本很低,研究表明物理键盘上打字的肌肉记忆能够较好进行迁移\cite{palmboard2020}\cite{2018shitoast}。此外,因为输入媒介不是触摸屏,用户在输入间隙时,可以将双手放在桌面上休息
    \item 充分利用了手机的信息传感能力,提出了一套基于深度摄像头数据和声音数据进行点击识别的流程及对应算法
    \item 在虚拟键盘上集成了单词级别输入和字符级别输入两种输入模型,弥补了较多文献中的不足,同时也更能够满足用户的日常生活需要
\end{itemize}

\section{文章结构}
文章主体部分一共包括六个章节,其中本章简要介绍了任意桌面文本输入工作的动机和主要内容。第二章则系统介绍了相关工作,包括任意桌面文本输入的研究,使用手机作为交互的输入方式,字符级别输入的研究以及常用的纠错算法等几个方面。第三章介绍了第一个用户实验,采集用户在桌面上十指打字的点击数据以及相应结果,包括落点分布、输入速度、键盘布局等内容。第三章分为两个部分,第一部分详细阐述了在本文的工作中,如何使用声音、深度图像多种传感通道识别用户的点击,并且计算出对应点击的三维坐标;第二部分则描述了如何在实验一数据的基础上对各种用户意图推理算法进行模拟实验,探究合适的算法。第五章介绍了第二个用户实验,从单词级别输入、字符级别输入和混合模式输入三种模式评测本文所提出的输入技术。最后一章对本项工作进行了总结,并指出了可以深入扩展的部分。