% !TeX root = ../main.tex
\chapter{文献综述}
\label{cha:literature}
%前言
在这一章中,我们首先回顾了在较大输入空间进行十指打字的研究,包括大尺寸触摸屏、桌面等,对研究中用户的行为、点击的识别等问题进行了讨论。其次,我们探索了使用手机作为交互方式的文本输入,包括直接和间接的输入方式。同时,文章中也简要回顾了几种字符级别输入的设计。然后,文字讨论了文本输入研究中使用最为广泛的贝叶斯算法及其改进。最后,对前人工作进行总结并指出我们工作所在的位置。

\section{任意桌面文本输入} %投影键盘,空中打字,tiptext
随着各种尺寸的触摸屏出现之后,研究者就一直在试图探究人们在各种平面上的打字行为。对于物理键盘而言,十指打字是绝大部分用户的使用方法。当形成肌肉记忆时,用户可以做到不看键盘盲打,人们的平均打字速度能够达到30-40单词每分钟,熟练的用户能够达到60单词每分钟\cite{flatglass2011findlater}。然而,在商业产品中,使用同样QWERTY布局、尺寸相似的虚拟键盘,例如iPad和Microsoft Surface上的键盘,输入速度却和物理键盘相差甚远。但是同时,虚拟键盘主要基于软件,为各种动态调整的算法提供了使用空间。因此,众多前人的工作都在研究用户在缺乏反馈的平面或者是虚拟键盘上的输入行为。

Mackenzie通过对比研究表明\cite{mackenzie2001empirical},同样使用QWERTY布局,物理键盘上的打字速度和软键盘的输入速度存在正相关性,说明打字的熟练程度存在转移的可能性。在平面上,Findlater针对是否显示键盘以及是否有错误反馈下的不同情况进行了对应的实验和数据分析\cite{flatglass2011findlater}。实验结果表明,当用户在没有任何限制的情况下输入时,能达到59单词每分钟的输入速度,这是理论上使用双手在平面上输入的速度上限。对于不同的用户而言,拟合出的按键大小和形状也存在差异,证明了适应性算法的必要性。

针对用户手的位置不固定的情况,TOAST\cite{2018shitoast}则在之前的基础上进行了进一步探索。结果显示,拟合出的键盘随着时间的推移不断变大,两只手会逐渐分开,键盘呈现一定的弧度,但是整体上趋近于矩形布局。通过合适的动态调整的算法,TOAST\cite{2018shitoast}最终也在平面上实现了和物理键盘使用方法相似这一十分自然的输入方式,最终能够达到43.1单词每分钟的输入速度,同时也验证了速度和物理键盘速度相关性的结论\cite{mackenzie2001empirical}。

%murase2012gesture machine learning, gesture

如何区分有意和无意点击也是在虚拟键盘上输入的难点之一,常见的误触包括手掌的放置以及手指无意间触碰等。防手掌误触可通过触碰的位置\cite{2018shitoast},分析手掌图像\cite{ewerling2012finger},收集点击信息并进行分类\cite{schwarz2014probabilistic}等方式进行区分。对于手指的无意碰触,可以结合具体使用场景通过力度大小、持续时间等信息进行判断。

除了与触摸屏等电子平面进行交互之外,直接在桌面上进行输入更为方便,受到场景限制的可能性更小,例如通过红外光在桌面投影的Canesta键盘\cite{roeber2003typing},或者是只使用摄像头并采用机器学习方法识别\cite{murase2012gesture}。在特定条件下,指尖\cite{2019tiptext}、半空\cite{2015atk}都可以作为输入平面。

\section{基于手机的文本输入} %pressure
智能手机如今已经成为人们日常生活和移动办公中不可或缺的一员。手机的便携性和多传感通道的优势为人机交互开创了前所未有的空间,许多最新的研究都围绕手机展开。例如基于手机摄像头和麦克风识别交互手势的ProxiTalk\cite{yang2019proxitalk},借助三棱镜扩展手机摄像头视野从而追踪手部行为的HandSee\cite{yu2019handsee},借助耳朵与手机交互的EarTouch\cite{wang2019eartouch}等等。文本输入毫无疑问也是人们在使用手机时的重要需求之一,众多工作以手机为载体,从不同的角度探索如何获得高效便捷的输入体验。

类似在平板上的研究思路,如何在使用手机输入时解放双眼一直是研究重点之一。BlindType\cite{2017blindtype}研究了在手机上使用拇指输入,在大屏幕上显示结果的间接输入情况。尽管不同按键有重叠,落点拟合出的键盘仍然符合QWERTY布局,说明用户仍然能够将自己的肌肉记忆转移到移动设备上。进一步的研究发现,在手机上不显示键盘的直接输入方式同样能够利用之前的肌肉记忆\cite{zhu2018typing}。

除了使用电容屏的点击作为交互途径之外,其他的信息通道能够为文本输入带来新的功能以及交互体验。压力也是常见的利用方式,使用压力可以控制字母光标的移动\cite{2018forceboard},调节点击的不确定性\cite{weir2014uncertain},切换字母大小写\cite{brewster2009pressure}等。


% 盲人在手机上的文本输入

前述的工作表明,在较大的平面上使用十指打字能够达到较快的输入速度\cite{2018shitoast},而交互区域较小时,输入时一般采用部分手指进行点击\cite{2017blindtype}\cite{zhu2018typing}。除了点击,也有一些技术使用其他的交互方式进行输入,例如手势输入\cite{murase2012gesture}\cite{zhu2019sfree},移动光标选择\cite{2018forceboard}等。

\section{字符级别输入}
在现实生活中,字符级别的输入也是不可缺少的功能。字符级别的输入区别于单词级别输入,后者默认用户的一连串点击对应了词库中的某个单词,因此通常能够使用语言模型进行优化。然而,字符级别用户输入的不同字符之前几乎没有关联,例如用户在输入验证码和密码的使用场景。

如果显示区域足够大并且按键易于瞄准,显示完整键盘是最自然的字符输入方式。然而,很多使用场景并不满足该条件,具体的交互方式也存在一定差异。例如在一维空间中通过压力\cite{2018forceboard}或者手势\cite{1dhandwriting}控制光标进行选择,在二维触摸屏通过多点触控实现\cite{bonner2010no},或者是划分区域供用户选择\cite{banovic2013escape}。字符级别输入因为需要精准进行选择,因此输入速度通常较低,一般处于2WPM至5WPM之间。

\section{用户输入意图推理方法}
输入意图推理方法,指的是将用户的点击映射成字符或者单词的过程,对应字符级别的输入推理和单词级别的输入推理。目前大部分的研究都属于单词级别的输入,即将用户一连串的点击转换成词库中的单词。准确迅速地翻译出用户的输入意图,是文本输入研究的核心范畴,通常要结合基于具体的使用场景给出合理假设,并且提出对应的数学模型。Goodman\cite{language2002goodman}提出了朴素贝叶斯的方法,将其表示为数学形式为:
\begin{equation}
    \label{equ:argmax-word}
    \begin{aligned}
    argmax P(W|I) &= argmax \frac{P(I|W) \times P(W)}{P(I)} \\
                  &= argmax P(I|W) \times P(W)
    \end{aligned}
\end{equation}

其中,$I$指的是用户一连串的输入序列,$W$指的是词库中的一个单词,(\ref{equ:argmax-word})目的则是找到在输入序列为$I$的情况下,使得$P(W|I)$概率最大的单词。其中,$P(W)$是一个单词在文本库中使用到的频率,称为语言模型;$P(I|W)$则由用户的点击决定,称为输入模型或者点击模型。为了计算简便,我们假设每次点击是独立事件,且一共有$n$次点击,那么$P(I|W)$可写成:
\begin{equation}
    P(I|W) = \prod_{i=1}^{n}P(I_i|W_i)
\end{equation}

式子中,$I_i$表示第$i$次点击,$W_i$表示单词中的第$i$个字母。假设用户对每个按键的所有点击都分别满足高斯分布,那么$P(I_i|W_i)$可以使用对应的公式计算。例如当点击在二维空间中时,$P(I_i|W_i)$是两个维度高斯分布计算的概率乘积。尽管原理十分简洁易懂,多项研究\cite{2017blindtype}\cite{2015atk}\cite{2018forceboard}都显示了该方法的有效性。

在该方法中,当用户的打字指法遵循一定的规律时,例如用户使用的是标准指法\cite{2015atk},每次点击对应的按键会减少为三分类或者六分类,预测准确率能得到进一步提升。

朴素贝叶斯推理算法基于以下两个假设,一是每个按键的中心点是固定的,因此如果用户打字过程中手的位置发生偏移,潜在的键盘布局也会偏移,导致出现较大的误差;二是把每次点击看成是独立的,但是实际输入时两次连续的点击相对位置的偏移往往也遵循一定的规律\cite{2018shitoast}。TOAST\cite{2018shitoast}在朴素贝叶斯算法的基础上,引入了马尔可夫-贝叶斯的意图推理算法。该方法的主要思想是,将用户的点击看作是一个二阶马尔可夫模型。该方法解决了上文中提到的手的位置移动的问题,即使在不同时间段的点击中手发生了偏移,两次连续的点击相对位置的变化仍然是服从高斯分布的。表示为数学形式如下:
\begin{equation}
    \label{equ:markov-bayesian}
    P(I|W) = P(I_1|W_1) \times \prod_{i=1}^{n}P(I_i, I_{i+1}|W_i, W_{i+1})
\end{equation}

此外,用户在点击不同行时的持续时间,点击区域大小也会有差异。在分类时,可以使用这些信息区分用户点击的是键盘的哪一行,将其信息加入公式(\ref{equ:markov-bayesian})即有:
\begin{equation}
    \begin{aligned}
        P(W, \textbf{F}|W) &= P(I,\textbf{F}|W) \times P(W) \\ 
        &= P(I|W) \times P(\textbf{F}|W) \times P(W) \\ &= P(I|W) \times \prod_{i=1}^{n}P(\textbf{F}_{i}|W_{i}) \times P(W)
    \end{aligned} 
\end{equation}

其中,$\textbf{F}$表示持续时间等信息构成的向量,PalmBoard\cite{palmboard2020}的研究结果表明,在单手输入的情况下,利用这些信息能够将朴素贝叶斯算法的准确率提高超过20\%。除了增强点击模型,还可以改善语言模型,例如采用最小字符串距离搜索算法\cite{tinwala2010eyes},寻找单词最近邻\cite{bunke1993fast}等方法提高预测单词的准确率。当用户使用手势输入单词时,也同样可以用贝叶斯的原理推测用户输入\cite{zhai2012word}。

\section{文献总结}
本章从不同的角度对于文本输入的相关工作进行了一个总体和粗略的概括,可以看到很多工作都是基于单手盲打或者十指盲打展开的。

盲打是熟练用户使用物理键盘的输入方式,盲打解放了双眼,能极大提高输入速度和改善用户体验。因此,在使用虚拟键盘时,无论是在平板电脑还是手机上等,研究者都首先试图探索的都是能否将物理键盘上的肌肉记忆转移到对应的场景,多项研究结果都证实了这种可能性\cite{2017blindtype}\cite{2018shitoast}\cite{zhu2018typing}\cite{palmboard2020}。然而,由于缺乏反馈以及尺寸受限等原因,拟合出的键盘虽然通常符合QWERTY布局,但是会出现相邻按键重叠的情况。此时为了保证一定的输入的准确性,则需要探索用户在该场景下的使用特征,并且研究试验合适的算法模型。

在移动设备上的文本输入不仅要满足速度的要求,更需要考虑便携性、空间大小、使用简易程度等一系列问题。从而,除了使用物理键盘的布局辅以算法的研究思路之外,利用软件的灵活性提升移动设备的输入体验也是可以探索的重点之一,通常可以通过调整布局、改善交互方式等途径来实现。

在现有的文献中,有的技术具备在任意桌面上进行文本输入的能力,例如投影键盘等,但是其不仅需要特定的红外等设备支持,而且使用体验较差。同时,也更不乏利用手机的传感信息作为输入途径的技术,但是这些技术本身还是将手机作为输入媒介。根据我们的了解,还没有出现利用手机传感实现在任意桌面的文本输入的相关工作。本工作则是大大减小了设备本身尺寸等的影响,不仅充分利用了人们随身携带的手机上日益增强的传感功能,也具备了在任意桌面上进行输入的能力。另外,区别于之前的十指输入,本工作同时结合了字符输入,并且方便切换。总体上,我们的技术满足用户易用性和便携性的要求,填补了相关领域研究的空白。